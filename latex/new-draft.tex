\documentclass{article}
\title{Negation in Fichte's 1794 \textit{Foundation of Theoretical Knowledge}}
\author{Marcus Lampert  \\
	Unaffiliated\\
	}

\date{\today}
\begin{document}

\maketitle


\begin{abstract}
Short introduction to subject of the paper \ldots 
\end{abstract}

\section{Introduction}
This article is intended to serve as a roadmap for understanding Fichte's 1794 \textit{Foundation of Theoretical Knowledge}.
We interprete Fichte's text as a discussion about the role that negation plays in judgement and we show that, viewed in this way, the \textit{Foundation of Theoretical Knowledge} maintains a consistent and persistent argumentative line concerning the internality of negation to all judgement.
According to this main argumentative line, the negation of a judgement 'p', which we can designate as 'not-p', is not a judgement that is separate and independent from 'p'.
Rather, on the view Fichte wants to espouse, the judgement 'not-p' is already contained within, hence internal to, the judgement 'p'.
We show in the article that Fichte begins the \textit{Foundation of Theoretical Knowledge} by entertaining the hypothesis that 'not-p' is a separate and independent judgement from 'p', and then moves step by step to his final position, according to which negation, or 'not-p', is internal to the judgement 'p', and thus simply a different way of expressing or intending 'p'.
\footnote{This line of argument is greatly indebted to Irad Kimhi's thoughts on negation in \textit{Thinking and Being}.  Indeed, one could say that this article presents a Kimhian reading of Fichte's theoretical philosophy.  I believe that such a reading is successful because Fichte is as points trying to say something similar to what Kimhi says regarding the nature of judgement.}
\section{The problem of negation}
Before we look at argumentative roadmap that Fichte text from a theory of external negation to a theory of internal negation, we had to understand why negation is, in general, a topic of interest for Fichte. 
\section{Three Types of Negation}

\paragraph{Outline}
First we start with a little example of the article class, which is an 
important documentclass. But there would be other documentclasses like 
book \ref{book}, report \ref{report} and letter \ref{letter} which are 
described in Section \ref{documentclasses}. Finally, Section 
\ref{conclusions} gives the conclusions.



\section{Documentclasses} \label{documentclasses}

\begin{itemize}
\item article
\item book 
\item report 
\item letter 
\end{itemize}


\begin{enumerate}
\item article
\item book 
\item report 
\item letter 
\end{enumerate}

\begin{description}
\item[article\label{article}]{Article is \ldots}
\item[book\label{book}]{The book class \ldots}
\item[report\label{report}]{Report gives you \ldots}
\item[letter\label{letter}]{If you want to write a letter.}
\end{description}


\section{Conclusions}\label{conclusions}
There is no longer \LaTeX{} example which was written by \cite{doe}.


\begin{thebibliography}{9}
\bibitem[Doe]{doe} \emph{First and last \LaTeX{} example.},
John Doe 50 B.C. 
\end{thebibliography}

\end{document}

